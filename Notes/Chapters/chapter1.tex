\chapter{Introduction}

\clearpage
\section{What is a Project?}

	\begin{itemize}
		\item{\bf A process} refers to ongoing, day-to-day activities in which an 
		organization engages while producing goods or services.
		\item{\bf A project} is a unique venture with a beginning and end, conducted by people
		to meet established goals within parameters of cost, schedule, and quality. 
		\item{\bf Projects} are goal-oriented, involve the coordinated undertaking of 
		irrelated activities, are finite duration, and are all, to a degree, unique.
		\item{\bf A project can be considered to be any series of activities and tasks that:}
			\begin{itemize}
				\item Have a specific objective to be compleated withincertain specifications
				\item Have defined start and end dates
				\item Have funding limits (if applicable)
				\item Consume human and nonhuman resources (i.e., money, people, equipment)
				\item Are multifunctional(i.e., cut across several functional lines)
			\end{itemize} 
		\item{\bf A project is} organized work toward a predefined goal or objective that
		requires resources and effort, a unique (and therefore risky) venture having a 
		budget and schedule. 
		\item{\bf Elements of projects:}
			\begin{itemize}
				\item {\bf Projects are complex, one-time processes:} specific purpose or to meet
				a stated goal. Requires the coordinated inputs of numerous members of the 
				organization. Project members may be from different departments or other 
				organizational units or from one functional area. 
				\item {\bf Projects are limited by budget, schedule, and resources:} project 
				work requires that members work with limited financial and human resources for
				a specified time period.
				\item {\bf Projects are developed to resolve a clear goal or set of goals:} 
				There is no such ting as a project team with an ongoing, nonspecific purpose. 
				The project's goals, or deliverables, define the nature of the project and that
				of its team. 
				\item {\bf Projects are customer-focused:} the underlying purpose of any project
				is to satisfy customer needs. Projects were considered successful if they 
				attained technical, budgetary, or scheduling goals. More and more, however, 
				companies have realized that the primary goal of a project is customer 
				satisfaction.
			\end{itemize}
	\end{itemize}

\clearpage
\section{General Project Characteristics}
	\begin{itemize}
	\item {\bf Projects characterized by the following properties:}
		\begin{enumerate}
			\item {\bf Projects are ad hoc endeavors with a clear life cycle}
			\item {\bf Projects are building blocks in the design and execution of organizational 
			strategies}
			\item {\bf Projects are responsible for the newest and most improved products,
			services, and organizational processes}
			\item {\bf Projects provide a philosophy and a strategy for the management of change.}
			Successful organizations routinely ask for customer input and feedback to better 
			understand their likes and dislikes.
			\item {\bf Project management entails crossing functional and organizational 
			boundaries:} A project aimed at new product development may require the combined 
			work for engineering, finance, marketing, design, and so forth. 
			\item {\bf The traditional management functions of planning, organizaing, motivation, 
			directing, and control apply to project management.} The project manager is the 
			person most responsible for keeping track of the big picture.  
			\item {\bf The principal outcomes of a project are the satisfaction of customer
			requirements within the consstraints of technical, cost, and schedule objectives.}
			Projects are defined by their limitations. They have finite budgets, definite
			schedules, and carefully stated specifications for completion.
			\item {\bf Projects are terminated upon successful completion of performance
			objectives}
		\end{enumerate}
	\item Projects differ from better-known organizational activities, which often involve 
	repetitive processes. The traditional model of most firms views organizational activities
	as consistently peforming a discrete set of activities. For example, a retail-clothing 
	establishment buys socks, and sells clothes ina continuous cycle. 
	\item {\bf Process vs. project:}
		\begin{table}[H]
			\centering
			\begin{tabular}{| l | l |}
				\hline
				Process & Project \\ \hline
				Repeat process or product & New process or product \\ \hline
				Several objectives & One objective \\ \hline
				Ongoing & One shot - limited life \\ \hline
				People are homogenous & More heterogeneous \\ \hline
				Well-established systems in place to integrate efforts & Systems must be
				created to integrate efforts \\ \hline
				Greater certainty of performance, cost, schedule & Greater uncertainty
				of performance, cost, schedule \\ \hline
				Part of line organization & Outside of line organization \\ \hline
				Bastions of establishment practice & Violates established practice \\ \hline
				Supports status quo & Upsets status quo \\ \hline
			\end{tabular}
		\end{table}
	\end{itemize}

\clearpage
\section{Why are Projects Important?}
	\begin{itemize}
		\item {\bf Shortened product life cycles:} the days when a company could offer a 
		new product and depend on having years of competitive domination are gone.
		Increasingly, the life cycle of new products is measured in terms of months or
		even weeks, rather than years. One has only to look at new products in electronics
		or computer hardware and software to onnserve this trend. 
		\item {\bf Narrow product launch windows:} Organizations are aware of the dangers
		of missing the optimum point at which to launch a new product and must take a proactive
		view toward the timing of product introductions. For examples, while reaping the profits
		from the successful sale of product A, smart firms are already plotting the best
		point at which to launch product B, either as a product upgrade or a new offering.
		\item {\bf Increasingly complex and technical products:} we want the new models of
		our consumer goods to be better, bigger (or smaller), faster, and more complex than 
		the old ones. 
		\item {\bf Emergence of global markets:} The increased globalization of the economy,
		coupled with enchanced methods for quickly interacting with customers and suppliers,
		has created a new set of challenges for business. These challanges also encompass
		unique opportunities for those firms that can quickly adjust to this new reality.
		\item {\bf An economic period marked by low inflation}
	\end{itemize}

\section{Project Life Cycles}

	A {\bf project life cycle} refers to the stages in a project's development:
	\begin{itemize}
		\item {\bf Conceptualization}
		\item {\bf Planning}
		\item {\bf Execution}
		\item {\bf Termination}
	\end{itemize}

\section{Determinations of Project Success}

\section{DEveloping Project Management Maturity}

\section{Project Elements and Text Organizations}

