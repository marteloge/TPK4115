\section{Teori}

{\bf OBS, denne delen er ikke en del av besvarelsen.}

{\bf Resultatmål (også kalt kun mål)} beskriver hva et prosjekt eller tiltak skal oppnå og er knyttet 
til prosjektets resultater og leveranser. {\bf \color{red} Resultatmål. Det konkrete resultatet 
som skal leveres i prosjektet.}


{\bf Effektmålene (også kalt formål)} beskriver hvorfor prosjektet er etablert, og beskriver ofte en 
ønsket fremtidig situasjon som skal oppnås ved å gjennomføre prosjektet (målsetting).
Tenkemåten skal gjøre det lettere å planlegge større og eller langsiktige prosesser, der en gjerne 
deler disse opp i mindre deler, med egne delmål. Ofte vil slike prosesser lettes ved at det arbeides
 parallelt, men slik at delprosessene samles etterhvert for å oppnå hovedmålsettingen.
 (Fremtidige gevinster).

{\bf Interessenter}
Individ/gruppe/organ som kan påvirke eller bli 
berørt av prosjektet
	\begin{itemize}
		\item Eier 
				\begin{itemize}
					\item prosjektets oppdragsgiver som initierer eller bestiller prosjektet 
				\end{itemize}
		\item Bruker 
			\begin{itemize}
				\item anvender prosjektresultatet
			\end{itemize}
		\item Prosjektorganisasjon 
			\begin{itemize}
				\item ansvarlig for selve gjennomføringen av prosjektet etter bestilling fra eieren 
			\end{itemize}
		\item Andre 
			\begin{itemize}
				\item offentlige myndigheter, allmennheten,  interesseorganisasjoner og media.
			\end{itemize}
	\end{itemize}
