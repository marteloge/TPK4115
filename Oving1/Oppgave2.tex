\section{Oppgave 2}

{\bf Vil du vurdere prosjektet som en suksess eller fiasko? 
Gi begrunnelse for vurdering.}

{\bf Kritiske hendelser for vurdering av prosjektet:}

	\begin{itemize}
		\item {\bf Tidspress:} det ble satt et politisk press for å få
		igangsatt mudringen innen utgangen av 2009 for å vise at sysselsettingen 
		kom raskt i gang. Selve mudringen ble ikke satt i gang før april 2010, noe
		som var et halvt år etter planlagt oppstart. Dette er en faktor som gjør
		at effekten av prosjektet ikke blir synlig før mye senere, noe som kan påvirke
		andre beslutningsprosesser som avhenger av å se effekten av prosjektet.
		\item {\bf Økonomi:} prosjektet ble mer enn dobbelt så dyrt. 
		\item {\bf Fullføring av prosjekt:} selve prosjektet ble gjennomført, men 
		ikke alle resultatene ble like gode. Når det kommer til forurensningsmyndighetene
		ønsket de å rydde opp forurensningene i hele havneområdet, noe det ikke ble tid til
		på grunn av tidspress. Havna fremstår som delvis ren.
		\item {\bf Interessenter:} på grunn av uventede hendelser ble det dratt inn
		flere interessenter enn planlagt. 
	\end{itemize}

{\bf Status og resultater}
	\begin{itemize}
		\item {Havna har fått utvidet innseiling og ønsket seilingsdybde}
		\item Havna fremstår som renere
		\item Mudringen var estimert til å vare i 120 kalenderdager, men det tok omtrent dobbelt
		så lang tid.
		\item Kostnadene er var anslått til ca 70 millioner, noe som er over dobbelt så mye som var 
		estimert. 
		\item Interessenter som de som bruker havna, det lokale havnevesenet, kommunen, skippere
		ved ulike fartøy som gjerster havna og andre er kjempefornøyde med at havna er blitt enklere
		å manøvrere fartøyene i.
		\item Flere synes at havna har blitt renere
	\end{itemize}

{\bf Konklusjon:}

	