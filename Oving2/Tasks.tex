\section*{Oppgave 1}

	{\bf Bradley led som følger av problemer med uklart omfang, kontinuerlige revurderinger
	og forandringer i de opprinnelige spesifikasjonene. Hvilke direkte påvirkninger fikk
	forandringene i omfanget på det endelige designresultatet til Bradley?} \\

	{\bf Bakgrunn for design av Bradley:} ble planlagt og designet for å ta over for hærens
	daværende PPK (Pansret Personellkjøretøy), M-113.

	{\bf Det orginale designet til Bradley:}
	\begin{enumerate}
		\item Kapasitet til å frakte en hel infanteritropp (12 soldater).
		\item Topphastighet tilstrekkelig for å kunne holde følge med andre pansrede kjøretøy.
		\item Kraftig pansring i sidene for å beskytte besetning og infanteri i kjøretøyet.
		\item Evne til å kunne kjøre gjennom vann (amfibøse evner).
		\item Minimalt med offensiv våpenstyrke (for å kunne bevege seg så lett som mulig).
	\end{enumerate}

	{\bf Sluttproduktet Bradley:}
	\begin{enumerate}
		\item Kapasitet til å transportere 6 soldater.
		\item Så tynn pansring at fiendtlig ild lett penetrerte veggene.
		\item Sank som en sten ved forsøk på å krysse elver.
		\item Var fullt utstyrt med maskingevær, en 25mm kanon, og antitank-missiler.
	\end{enumerate}

\clearpage
	{\bf Hendelser frem mot sluttresultatet:}
	\begin{itemize}
		\item {\bf Uenigheter intert i hæren:} fra begynnelsen av var det interne krefter 
		i hæren som ikke klarte å bli enige om hvilken rolle Bradley skulle fylle.
		\item {\bf Bruksområde:} et fartøy i hæren bær kunne utføre en oppgave, men oppgaven
		skal den kunne utføre bra. Et problem oppstod når storkarene i hæren ville ha den til å 
		(1) avveksle den gamle PPK'en, (2) bruke den som pansret oppklaring. Disse oppgavene er
		oppgaver som ikke går hånd i hånd.
		\item {\bf PPK (den nye Bradley) vs. Pansret oppklaring:} PPK'en skulle ha god pansring, god 
		plass til å transportere soldater, og kun våpenstyrke til forsvar. I en pansret oppklaring 
		trengs det høy hastighet, samtidig som den skal utføre oppgaver av mer offensiv art. Disse
		motstridende kravene førte til at pansringen ble lettere og dermed av dårligere beskyttelse, 
		samt en rekke våpenstyrker som kanontårn, maskingevær og antitank-missiler.
		\item {\bf Tidsperspektivet:} prosjektet strakk seg over en alt for lang tidsperiode. Dette
		medfører en rekke personer inn og ut, samt utallige nye tegninger, noe som aldri fører til et
		endelig resultat. 
		\item {\bf Kvalitetskontroll:} under produksjonen ble det gjennomført dårlig kvalitetskontroll 
		av type: juks av testresultater, forfalsking av dokumentasjon og dårlig kvalitetskontroll på'
		produksjonslinjen. Det ble på grunnlag av dette funnet mangler, noe som førte til hurtige
		og lite gjennomtenkte besluttninger som førte til katastrofe etter katastrofe.
		\item {\bf Resultatet:} var et kjøretøy som hverken fungerer bra til det ene eller det andre.
	\end{itemize}

	Ved et meget uklart omfang, kontinuerlige revideringer og forandringer i de opprinnelige spesifikasjonene
	førte det til at sluttproduktet ble noe helt annet enn det i utgangspunktet var tenkt. 
	Sluttproduktet endte opp som en dårlig blanding av to verdener, noe som gir et kjøretøy som hverken
	kan utføre det ene eller det andre godt nok. I et vellykket prosjekt ville man kunne fanget opp
	de forskjellige vinklingene på hva fartøyet skulle gjøre og dermed fått frem to, men veldig forskjellige
	resultater som ville kunne utføre sine oppgaver til det ypperste.

\clearpage
\section*{Oppgave 2}
	
	{\bf Når er forandringer i omfanget til et prosjekt nyttig? Når er det farlig?}

		Endringer i et prosjekt er ofte knyttet til nye ønsker fra en kunde, erfaringer underveis
		eller kostander/tidsrammer. 

		Når det kommer til ønsker fra kunde er det ikke alltid en kunde vet hva de vil ha
		som et sluttresultat. Som prosjektleder er det viktig å ha et klart definert
		omfang som både prosjektteamet og kunder er enig i og forstår. Det at en kunde kommer med
		et nytt krav som er utenfor omfanget er somregel ikke greit. 
		Dette er noe vi kan kjenne igjen i scenarioet som er beskrevet i øvingen. Her var det for 
		det første fra starten av store uenigheter om hva som skulle leveres til slutt. 
		For det andre var det ingen retningslinjer på når endringer kunne bli vurdert og hvem
		som kunne komme med endringer. Det var konstante endringer, ofte kolliderende med eksisterende
		design, som førte til store utsettelser. 

		I prosjekter blir det ofte endringer i omfang på grunn av erfaringer som blir gjort underveis. 
		Når man ser at en besluttning man tok i starten går ut over sluttproduktet er det lurt å ta
		en endring i omfanget. For eksempel fra caset er det veldig nyttig å ta en revurdering på
		omfang når noen krav kolliderer med hverandre eller feks at materiale som er valgt ikke 
		fungerer som planlagt. Her vil det bli nyttig å få fikset opp i problemet så fort det blir
		oppdaget på det grunnlag av at det uansett må fikses på. 

		De fleste prosjekter har en økonomisk ramme å forholde seg innenfor. I noen situasjoner blir
		det nødvendig å endre omfang på det grunnlag av at økonomien ikke strekker til. I caset i 
		oppgaven sier de ikke stort om økonomiske rammer. Dette er nok en begrensning som burde blitt
		definert fordi et prosjekt styres av økonomi- og tidsrammer, mens dette prosjektet bare pågikk
		uten å ha en definert ende på tid eller penger. 
		Når det er økonomisk knapphet eller tidspress er det ofte vanlig å endre prioritering/varighet
		av spesifikasjoner eller fjerne noen helt. 

\clearpage
	{\bf Under hvilke omstendigheter kan en organisasjon nekte å fryse designspesifikasjonene
	til et prosjket av gyldige grunner?}

		Det finnes visse situasjoner der man bare må ta tak for å endre spesifikasjonene. 
		I utdelt case var det et godt eksempel der materialene som ble brukt utløste giftige gasser, 
		samt at det var av dårligere sikkerhet. Her står det mennesker på spill, noe som gjør at
		det er spesifikasjoner som MÅ endres på. 

		I noen omstendigheter strekker ikke midlene, tiden eller ressursene til. Dette er situasjoner
		som ikke kan ignoreres og må av den grunn kunne gå som en gyldig grunn til å endre på
		designspesifikasjonene. 

\clearpage
\section*{Oppgave 3}

	{\bf Anta at du ble hentet inn som en ekstern konsulent i forbindelse med utviklingen av Bradley
	i 1960 årene. Nevn noen av faresignalene som kunne vært observert i forbindelse med utviklingsproblemene:}
	\\


	{\bf Her er noen faresignaler som ville blitt obsertvert i forbindelse med utviklingsproblemene:}
		\begin{enumerate}
			\item {\bf Prosjektrammer:} i et prosjekt er det visse faktorer som må forhåndsbestemmes
			for å kunne gjennomføre et suksessfullt prosjekt: 
			\begin{enumerate}
				\item {\bf Økonomi:} det blir ikke henvist til noen økonomiske rammer i prosjektet, 
				noe som gjør det lettere å endre spesifikasjonene uten gyldig grunnlag. 
				\item {\bf Tid:} det virker ikke som om det er en klar plan på når prosjektet skal
				stå ferdig, noe som gjør det lett å bare endre spesifikasjonene uten gyldig grunnlag.
				\item {\bf Rollefordeling:} det ser ikke ut til å være klare rollefordelinger. Med roller
				peker det spesielt mot en lederrolle. Under omstendighetene virker det som om en gjeng
				med storkarer leker ledere med forskjellige tanker om sluttproduktet som drar prosjektet
				i forskjellige retninger.
			\end{enumerate}
			\item {\bf Avgjørelsegrunnlag:} det virker som om det er en del forskjellige personer uten 
			definerte roller i prosjektet som tar viktige avgjørelser uten samtykke fra resten av 
			interessentene. Dette kan lett føre til at prosjektet kan ta en drastisk avsporing fra
			omfanget som i utgangspunktet var definert. 
			\item {\bf Endringsgjennomgang:} når det kom opp en ønsket endring virker det som om de
			bare bad om endringen til designeren uten å diskutere endringen opp mot de orginale spesifikasjonene.
			Dette er et stort faresignal om at spesifikasjonene skal gå utenfor omganget som opprinnelig ble
			definert. 
			\item {\bf Kvalitetssikring:} i starten av prosjektet virker det ikke som om kvalitet og sikkring
			av produktet hadde noe fokus. Et slikt fagfelt er i og seg veldig vanskelig å gjennomføre
			på det grunnlag at man må ha klare spesifikasjoner.
		\end{enumerate}

\clearpage
\section*{Oppgave 4}
	{\bf Utvikle en strategi for å håntere de forskjellige interessentene i Bradley-prosjektet.
	Angående prosjektets omfang, kan du identifisere kompromisser?}


\clearpage
\section*{Oppgave 5}
