\section*{Oppgave 2}

	{\bf Hva ville svaret i oppgave 1 være hvis kontraktstørrelsen var på 50 millioner
	NOK og involverte utvikling, planlegging og produksjon av 60.000 enheter av medisinsk utstyr?
	Anta også at varigheten på kontrakten vil være omtrent 36 måneder, hvorav 6 måneder vil gå til
	utvikling, testing, innkjøp og produksjonsplanlegging. Produksjonsfasen er estimert til å vare
	i 30 måneder og krever 10 personer i måneden.} 

	{\bf Prosjektets styringsvariable}
		\begin{itemize}
			\item {\bf Arbeidsomfang:} 10 personer i mpneden i 30 måneder
			\item {\bf Tid:} 36 måneder
			\item {\bf Kostnad:} 50 millioner kr
			\item {\bf Produksjonsantall:} 60.000 enheter av medisinsk utstyr
		\end{itemize}

		Siden styringsvariablene er endret med ganske ekstreme tall i forhold til det forrige og det 
		skal produseres 60 000 enheter så er nok prosjektet ganske annerledes og mye større.
		I tillegg skal det ikke bare arbeid fra en av avdelingene som er nødvendig med tanke p at det er 
		flere arbeidsoppgaver som innkjøp og produksjonsplanlegging. Dette gjør også at prosjektet blir 
		litt todelt med avsluttende produksjonsfase. Kompleksiteten økes også siden underleverandørens 
		rolle må avgjøres mens prosjektet er i gang.

		Her kan det være lurt å velge en balansert matriseorgansasjon. Dette er for å få en ansvarlig 
		prosjektleder som leder alt arbeidet og som ikke blir forstyrret av andre oppgaver. Det er 
		arbeiderne i de ulike avdelingene som gjør arbeidet. I starten så vil FOU-avdelingen være den 
		viktigste sammen med de som driver med innkjøp og leveranse. Siden FOU har så mange medarbeidere 
		kan de mest sannsynlig jobbe med andre ting i tillegg. Produksjon og montasje vil etterhvert ta 
		over osm de mest engasjert i prosjektet. Siden dette er kun en avdeling som vil være aktuell i 
		denne prosessen så kan man igjen kjøre prosjektet som en funksjonell organisasjon.

		En matriseorganisasjon vil også kunne støtte kommunikasjonen mellom de forskjellige divisjonene. 
		Dette er viktig der flere er innblandet og at de er avhengig av hverandre for å få de forskjellige
		fasene til å komme i mål med tanke på at de er avhengige av hverandre. 

		En matrisstruktur er en sammensetning av funksjonell og prosjekt, men det er ofte vanlig å definere
		om matrisestrukturen er svak (vipper mot funksjonell) eller om den er sterk (vipper mot prosjekt).
		I dette tilfellet hvor det kreves et godt samarbeid mellom divisjonene vil det kreve at 
		strukturen er en sterk matrisestruktur. 

