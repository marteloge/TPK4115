\section*{Oppgave 1}

	{\bf Tatt i betraktning størrelse og type for denne kontrakten, hvilken organisasjonsform 
	tror du er mest hensiktsmessig for denne kontrakten?}

	{\bf Prosjektets styringsvariable}
		\begin{itemize}
			\item {\bf Arbeidsomfang:} 750 timer utvikling og testing
			\item {\bf Tid:} 6 måneder
			\item {\bf Kostnad:} 5 millioner kr
		\end{itemize}

		{\bf Prosjektets omfang:} 750 timer vi.l si at prosjektet sysselsetter nesten en full stilling i prosjektperioden
		(750/5/8 = 18.75 uker/4uker = ca 5 måneder fulltid 8-16 hver ukedag).
		Dette vil da bli et relativt lite prosjekt. 

		{\bf Ressurser:} prosjektet vil kun kreve arbeid fra FOU, og 
		den avdeligen består av hele 30 ansatte, så problemet er ikke særlig sammensatt.
		Prosjektet kan beregnes som standardleveranse, da det er innenfor rammene av hva bedriften 
		gjør til vanlig.

		{\bf Organisasjonsmodell:} Til dette prosjektet bør det velges en organisasjonsmodell med 
		kontrollerende autoritet. Da kan hovedansvaret legges hos ledern av FOU. Hvis det skulle 
		trengs ressurser fra andre avdelinger så vil FOU-avdelingens leder få tilgang til disse 
		ressursene gjennom lederne av de andre avdelingene. Det kreves med andre ord ikke ressurser
		fra flere divisjoner, noe som gjør at behovet for kommunikasjon på tvers etter denne kontrakten
		ikke blir nødvendig. 
		I boken blir det presentert 3 forskjellige organisasjonsformer: funksjonell, prosjekt og 
		matrise. 

		{\bf Konklusjon:} siden prosjektet kan gjennomføres innen en divisjon (FOU) og at prosjektet
		krever lite samarbeid på kryss av divisjonene vil vi foreslå en funksjonell organisasjonsstruktur.
		Argumentet for denne organisasjonsstrukturen er at kontrakten ikke vektlegger noe krav om 
		direkte komminikasjon på tvers av divisjonene, samt at det ikke krever de store ressursene
		av mennesker.
			