\section{Oppgave 1 - Jobbrotasjon}

	{\bf Risikofaktorer som kan spores tilbake til aktører i prosjekter omfatter blant 
	annet (manglende engasjement og vilje, manglende kompetanse, holdninger, tilgjengelighet,
	motivasjon osv.). Hvilke konkrete tiltak har prosjektorganisasjonen i gang satt for å
	takle denne kategori av risikofaktorer i dette prosjektet?}

	\subsection*{Problemstilling}
		Beriften ønsker å innføre jobbrotasjon i bedrfte på grunnlag av negative hendelser som følge
		av enside og monotone arbeidsoppgaver som har ført til konsekvsener som fysiske skader, lite-
		engasjement og høyt sykefravær i bedriften. 

		Jobbrotasjon har tidligere blitt innført, uten stor suksess på grunn av organisering
		og gjennomføringen av prosjektet. Det ble før gjennomført en rotasjonssyklus på 6 måneder
		der gruppene bestod av en gruppeleder på 17 medarbeidere. De 17 personene ble fordelt på 
		4 forskjellige arbeidsoppgaver. Denne gjennomføringen fikk konsekvenser som:
			\begin{itemize}
				\item Liten fleksibilitet mellom arbeidsoperasjonene.
				\item Begrenset kompetanse hos den enkelte og krevne opplæring av 
				alle medarbeiderene ved rotasjon
				\item Stor sårbarhet ved sykdom på grunn av begrenset mulighet ved 
				rotasjon på grunn av manglende kompetanse
				\item Stort press på gruppeleder som satt med all kompetanse og ansvar
			\end{itemize}
		Prosjektet satte bedriften og medarbeiderene i en ugunstig situasjon. 
		Bedfrifthelsetjenesten fra bedriften og andre vet at en ordning som jobbrotasjon 
		er et godt tiltak for i arbeidet med å redusere effektene av ensidige og monotone arbeidsoppgaver,
		men det viste seg at gjennomføringen ikke var god nok. 

	\clearpage
	\subsection*{Tiltak for håndering av risikofaktorer}

		{\bf Etablering av prosjektgruppe:} for å kunne sikre at prosjektet gikk i en riktig
		retning ble det opprettet en prosjektgruppe som skulle følge opp med jevne mellomrom for 
		å sikre fremdrift, justere akkrivitetene ved behov og gi nødvendig støtte til gruppeleder.
		Med en slik ressursgruppe er det lettere å følge opp på forskjellig områder for å sikre
		at prosjektet ikke skulle kjøres inn i en situasjon der prosjektet stod som en risikofaktor
		for bedriften eller medarbeiderene. Denne gruppen var også viktig å ha for å møte
		de negative holdningene til løsningen. Det at bedriften bruker slike ressurser gir gode
		signaler og kan gi bedre holdninger til å prøve å gjennomføre prosjektet på en god måte. 
		Sammensettingen av prosjektgruppa var tverrfaglig som dekket et vidt spekter av 
		ansvars- og kompetanseområder.

		{\bf Definere suksesskriterier:} for å kunne vite om prosjektet ble gjennomført på en god
		måte og kunne måle dens suksess var det viktig å definere hvilke målbare resultater man 
		skulle se på.

		{\bf Gradvis implementering:} for ikke få en for stor forandring ble prosjektet implementert
		gradvis ved at det først ble satt inn 7 og deretter 10stk. Dette var et tiltak som ble
		satt i gang for å ikke risikere at prosjektet gikk ut over hva bedriften skulle levere.
		Begge rundene med prosjektet ble kjørt ganske tett opp mot hverandre slik at gapet
		mellom dem ikke fikk negativ påvirkning for gjennomføringen. 

		{\bf Evaluering fra medarbeidere før og etter prosjektet:} det ble satt i gang en
		spørreundersøkelse for prosjektet og i etterkant som gir datagrunnlag til å evaluere 
		prosjektets resultat. 

		{\bf Informasjonsutveksling:} for å hindre at det var dårlig innføring av ordningen 
		satte de i gang flere møter for informasjon og oppfølging. 
		Ved en slik involvering og informasjonsutvekslong kan det ha hjulpet til med å 
		redusere motstand under opplæringen.

		{\bf Gunstid tidspunk:} tidspunktet prosjektet ble satt i gang var gunstig med tanke på
		at bedriften hadde ledige ressurser til å bruke til en skikkelig opplæring. I denne 
		perioden var det nedgang som frigjorde de ressursene de trengte til en skikkelig
		gjennomføring og oppfølging. 

		{\bf Frie tøyler:} det ble gitt "frie tøyler" på hvordan prosjektet skulle utføres, 
		noe som gjorde at man kunne gjøre det på en måte som passet aktørene godt. 

		{\bf Helsemessig gevinst:} 	De sørget for å involvere fysioterapeaut for å vise fram 
		den helsemessige gevinsten ved jobbrtasjon. Dette var viktig for å overbevise 
		medarbeiderne.