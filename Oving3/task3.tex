\section{Utvikling av risikoreduserende strategier}

	{\bf Problem 3: utvikle en foreløpig risikoreduserende strategi for hver av de risikofaktorene 
	som er identifisert i problem 2. Hvis du måtte prioritere, hvilke risikofaktorer ville du 
	gjort noe med først? hvorfor?}

	\subsection{Oppsummering problem 2}
	\begin{table}[H]
		\begin{tabular}{ p{9cm} l l l}
			\hline
			Identifiserte risikofaktorer & Sann. & Kons. & Risiko \\ \hline
			{\bf 1.} Sentrale prosjektmedlemmer trekkes fra prosjektet & H & H & H \\
			{\bf 2.} Endring av økonomiske nedgangstider & L & H & M \\
			{\bf 3.} Kutt av prosjektmidler & M & M & M \\
			{\bf 4.} Endring i prosjektomfang (scope) & H & M & H \\
			{\bf 5.} Dårlig ytelse av spesifikasjon & L & M & L \\
			\hline
		\end{tabular}
	\end{table}

	\subsection{Prioritering av risikofaktorer}

		Det vi mener måtte blitt prioritert først er {\bf 4.} endring i prosjektomfang. Grunnen 
		til det er at til lengre man venter med endringer, i større grav vil endringene påvirke
		andre ledd i prosjektet. Denne er også rangert som en stor risiko (H x M = H).
		Som man ser i oppsummeringen ser vi også at sannsynlighet for endring i prosjektomfang er
		stor, noe som gjør at dette må bli satt som høy prioritet. 

		Som en god nummer 2 ville jeg prioritert {\bf 1.} sentrale prosjektmedlemmer trekkes fra 
		prosjektet. Når man trekker sentrale personer fra et prosjekt påvirker dette alle
		andre prosjektmedlemmer. Her vil alle andre måtte dekke en annen persons oppgaver, 
		i verste tilfelle er det ingen som kan ta over oppgavne, noe som fører til forsinkelser.

		Som delt 3. plass blir økonomi prioriter, der risikofaktor {\bf 2. og 3.} inngår. 
		Det er fordi dette er hendelser som til en viss grad kan ordnes hvis man er proaktiv.
		At det er nedgangstider er ikke noe man direkte kan påvirke, men man kan forberede seg
		på det slik at det ikke går ut over prosjektet. 

		Til sist inngår punkt 5, men er på det grunnlag at vi ikke helt forstod hva som 
		inngikk inn under dette punktet. 

	\clearpage
	\subsection{Risikoreduserende strategier}

		{\bf Risikoreduserende strategier:} Aksepter risiko, Minimer risiko, Del risiko,
		Overfør risiko, Sikkerhetsavsetninger, Opplæring og kompetanse 

		\begin{table}[H]
		\begin{tabular}{ c p{10cm} }
			\hline
			{\bf Risikofaktor} & {\bf Risikoreduserende strategi} \\ \hline
			{\bf 1.} & I en slik setting hvor prosjektmedlemmer trekkes fra et prosjekt
			er det viktig at man har andre i bedriften/prosjektgruppa som har kompetanse
			til å overta arbeidet. Her vil {\bf opplæring og kompetanseoverføring} ved for eksempel
			mentorordninger fungere for å risikere å ikke ha den rette kompetansen når et 
			medlem forsvinner. \\ \hline
			{\bf 2.} & Ved økonomiske nedgangstider kan et prosjekt bli stoppet av for lite
			ressurser i form av penger. Ved slike tilfeller kan det være lurt å sette av midler
			i form av {\bf sikkerhetsavsetninger}. At et byggeprosjekt stopper opp kan få store 
			konsekvenser. Ved å ha et reservelager gjør det at prosjektet kan unngå byggestopp
			og "låne" penger til økonomien går oppover. En slik risiko er vanskelig å gjøre noe
			med, men man kan gjøre det beste ut av det med et reservelager som en midlertidig løsning. 
			Sikkerhetsavsetning kan også gå under strategien {\bf minimer risiko}, der man minimerer
			konsekvenser som byggestopp ved å ha klar ekstra midler ved nedgangstider. \\ \hline
			{\bf 3.} & Kutt av prosjektmidler påvirker ofte alle ledd i et prosjekt. Dette fører
			ofte til omstruktureringer og endringer i planer. Her er det viktig at man {\bf minimerer
			risikoen} ved å ha god opplæring av alle ansatte og en reserveplan om det gjøres 
			budsjettkutt. \\ \hline
			{\bf 4.} & Hvis det gjøres endringer i et prosjekts omfang er det lurt å kunne {\bf overføre
			risiko}. Det kan gjøres ved at man har kjøpt et prosjekt med fastpris. Ved store endringer
			på grunn av eksterne faktorer eller interne fra leverandør kan man sikre seg mot store
			utforutsette utgifter. \\ \hline
			{\bf 5.} & \\
			\hline
		\end{tabular}
	\end{table}
